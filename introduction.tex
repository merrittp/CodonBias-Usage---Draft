\documentclass[../main.tex]{subfiles}
\begin{document}

\section{Introduction}

Mitochondria, despite maintaining a life-critical role in cellular function, have proven to host a great deal of genetic and functional diversity. Their proposed, and generally accepted, origin as a bacterial endosymbiont gives them a long and complex history with the eukaryotic cell. The vast majority of protein-coding sequences necessary for mitochondrial function have moved to the nuclear genome, yet the genome still codes for a handful of critical proteins.

 As such, the mitochondrial genome also contains its own set of transfer RNAs (tRNAs). The difference between mitochondrial tRNAs and their nuclear counterparts is poorly understood, though structural differences between the two types are well documented. In the majority of animal species, functional mitochondria code for a complete, though minimal, set of mitochondrial tRNAs, with each amino acid represented by their most degenerative codon. This complete set is deemed essential for human health, as many mitochondrial diseases seen in humans are a byproduct of missing or mutated mitochondrial tRNAs. 

Within Metazoa, there are exceptions to this observation, particularly in non-bilaterian animals, such as ctenophores, sponges, and cnidarians. In these lineages, mt-tRNAs have been independently lost multiple times. Ctenophores as a whole have lost all mt-tRNAs, and cnidarians all but two - methionine and tryptophan. Sponges show species-specific loss, where even closely related species in the same genus can show wildly different sets of mt-tRNAs. How this loss impacts organisms and how they compensate is unknown, though studies in plants and fungi report an unknown import mechanism could bring cytosolic tRNAs into the mitochondria. Previous studies from our lab report a correlation between the loss of mt-tRNAs and the loss of associated nuclear-encoded mitochondrial aminoacyl-tRNA synthetases and other modifying enzymes, showing that loss of mt-tRNAs has a detectable impact on both genome and protein evolution. 

\end{document}