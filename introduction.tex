\documentclass[../main.tex]{subfiles}
\begin{document}

\section{Introduction}
Here will be the background info about the organisms, sponge
mitochondria, necessity of looking at tRNAs and missing tRNAs, the questions about how do sponges compensate for the lack of certain tRNAs, and hypothesizing about codon bias and usage as a way of compensation.

Goals of this paper:
\begin{itemize}
    \item Report the generation and annotation of new demosponge mitochondrial genomes
    \item Detail the lack of codon bias in these genomes missing mitochondrial tRNAs when compared to species with complete tRNA sets
\end{itemize}

Talk about mitochondria, "mitochondria is the powerhouse of the cell"

Talk about tRNAs in mitochondria; full set of mitochondrial tRNAs in bilaterian animals. In non-bilaterian animals, independent loss of individual mitochondrial tRNAs in a variety of different clades.

Missing mitochondrial tRNAs - loss seems to be independent and species specific. Amphimedon queenslandica - missing 7. Plants, fungi, yeast, some other metazoans. Prominent in Porifera, Cnidaria, and Ctenophores, etc.

Why is studying missing tRNAs important

Compensation mechanisms to cope with missing tRNAs? Is codon bias and change in codon usage a way for mitochondria to cope with their missing tRNAS? Has there been convergent evolution of the mitochondria genome in response to tRNA loss? 

Generation of new mitochondrial genomes in these non-bilateral species will help evaluate compensation patterns, if any. 

The answer is no. There does not appear to be any bias in the new mitochondrial genomes that the genetic code/codon usage changes in response to missing tRNAS.

\subsection{Actual Introduction}

While many of the mechanisms contributing to mitochondrial replication and transcription are considered derived and might have origins in horizontal gene transfer events, mitochondrial translation mechanisms appear to retain their proposed bacterial origins. As part of this process, the mitochondrial genome also codes for transfer RNAs. The difference between these mitochondrial tRNAs and their nuclear counterparts is poorly understood, but in the majority of animal species, functional mitochondria code for a complete set of mitochondrial tRNAs. Indeed, this complete set is deemed essential for human health, as many mitochondrial diseases seen in humans are a byproduct of missing or mutated mitochondrial tRNAs. 

Within Metazoa, there are exceptions to this observation, particularly in non-bilaterian animals, such as ctenophores, sponges, and cnidarians. In these lineages, mt-tRNAs have been independently lost multiple times and across multiple groups. Ctenophores as a whole have lost all mt-tRNAs, and cnidarians all but two - methionine and tryptophan. Sponges show lineage-specific loss, indicating tRNA loss has occurred repeatedly and independently through time. How this loss impacts the organism and how they compensate is unknown, though studies in plants and fungi report an unknown import mechanism could bring cytosolic tRNAs into the mitochondria. If this is happening in metazoans has not been studied. However, previous studies from our lab report a correlation between the loss of mt-tRNAs and the loss of associated nuclear-encoded mitochondrial aminoacyl-tRNA synthetases and other modifying enzymes, showing that loss of mt-tRNAs has a detectable impact on both genome and protein evolution. 




\end{document}