\documentclass[12pt]{article}
\usepackage[utf8]{inputenc}
\usepackage[a4paper, total={6in, 8in}]{geometry}
\usepackage{parskip}

\usepackage{subfiles}

\title{Codon Bias and Usage - Draft}
\author{Merritt C. Polomsky}
\date{Spring 2020}

\begin{document}

\maketitle

\section{Introduction}
Here will be the background info about the organisms, sponge
mitochondria, necessity of looking at tRNAs and missing tRNAs, the questions about how do sponges compensate for the lack of certain tRNAs, and hypothesizing about codon bias and usage as a way of compensation.

Goals of this paper:
\begin{itemize}
    \item Report the generation and annotation of 3 new demosponge mitochondrial genomes
    \item Detail the lack of codon bias in these genomes missing mitochondrial tRNAs when compared to species with complete tRNA sets
    \item Look cool, cause we can.
\end{itemize}

Talk about mitochondria, "mitochondria is the powerhouse of the cell"

Talk about tRNAs in mitochondria; full set of mitochondrial tRNAs in bilaterian animals. In non-bilaterian animals, independent loss of individual mitochondrial tRNAs in a variety of different clades.

Missing mitochondrial tRNAs - loss seems to be independent and species specific. Amphimedon queenslandica - missing 7. Plants, fungi, yeast, some metazoans. Prominent in Porifera, Cnidaria, and Ctenophores, etc.

Why is studying missing tRNAs important

Compensation mechanisms to cope with missing tRNAs? Is codon bias and change in codon usage a way for mitochondria to cope with their missing tRNAS? Has there been convergent evolution of the mitochondria genome in response to tRNA loss? 

Generation of new mitochondrial genomes in these non-bilateral species will help evaluate compensation patterns, if any. 

The answer is no. There does not appear to be any bias in the new mitochondrial genomes that the genetic code/codon usage changes in response to missing tRNAS.

\subfile{methods.tex}
\subfile{results.tex}

\end{document}
